\chapter{Array 1-D, 2-D, 3-D}
\section{Intro}
	Array is like a storage, it can fill with string or integer. In 1-D, 2-D it can also represents the x-axis and y axis.
\section{Creating arrays}
	Arrays is created buy blanket.\\ \ \\
\noindent Example:\\
\begin{verbatim}
	a=[ ]        *a is the array name that you want.
\end{verbatim}
The things in the [ ] and be store and when you want to access it you will need its position in the array and type like  a[0]\\ \ \\
\noindent Example:\\
\begin{verbatim}
	a=["apple","orange","banana"]
\end{verbatim}
If you want to print banana form the array, you may want to type\\
\begin{verbatim}
print a[2] 
\end{verbatim}
\section{Filling arrays}
	Everything can be store in the array, strings, integers, arrays. When you create an array you can fill things in it as the default  things that the array have.\\ \ \\
\noindent Example:
\begin{verbatim}
    a=["Billy","Bud",90,60,50]
    b=["Anne","Chow",90,95,100]
    c=["Jen","Bo",60,80,90]
\end{verbatim}

If you want to add things into the array that u create already, you can use \\ \ \\

\begin{verbatim}
     array_name=array_name+[Things you want to add]
\end{verbatim}

\noindent Example:\\ \ \\

\begin{verbatim}	
    a=[apple]
\end{verbatim}

\noindent and now I want to add orange into it, so we add\\ \ \\

\begin{verbatim}
    a=a+[orange]
\end{verbatim}	

\noindent To create 2-D or more array we need to create array in the nested for-loop.\\ \ \\
\noindent Example:
\begin{verbatim}
a=[]
for i in range(N):	*N how long you want the array to be
    b=[]    *This is a temporary array to generate every array inside the main array.
    for j in range(N):
        *Things you want to put in the array by b=b+[ ]
    a=a+[b]	    *Here put the temporary array back to the main array.
\end{verbatim}		


\section{Traversing array}
Traversing array is visiting each element in the array and do something. In 1-D we can do it with for loop to identify things in array.
Example:

\begin{verbatim}
a=[1,2,3]
for i in range(len(a))    *len(a) =  Numbers of elements in the array
    *Things put here can edit the specific element a[i]
\end{verbatim}
	
\noindent In 2-D we start using nested for-loop to identify the x-axis and y-axis. So we use nested for-loop to traversing it too.\\ \ \\
\noindent Example:
\begin{verbatim}
a=[[0,1],[0,0],[0,1]]
for i in range(len(a)):
    for j in range(len(a)):
        *Things put here can edit the specific element a[i][j]
\end{verbatim}
			
\noindent In 3-D we use more for-loop to identify the more dimension.\\ \ \\
\noindent Example:
\begin{verbatim}
a=[[[0,0],[0,0]],[[0,0],[0,0]]]
for x in range(len(a)):
    for y in range(len(a)):
        for z in range(len(a)):
            *Things put here can edit the specific element a[x][y][z]
\end{verbatim}
	
	
	
	
	
	
